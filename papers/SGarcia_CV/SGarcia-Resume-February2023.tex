%%%%%%%%%%%%%%%%%%%%%%%%%%%%%%%%%%%%%%%%%
% Medium Length Professional CV
% LaTeX Template
% Version 2.0 (8/5/13)
%
% This template has been downloaded from:
% http://www.LaTeXTemplates.com
%
% Original author:
% Trey Hunner (http://www.treyhunner.com/)
%
% Important note:
% This template requires the resume.cls file to be in the same directory as the
% .tex file. The resume.cls file provides the resume style used for structuring the
% document.
%
%%%%%%%%%%%%%%%%%%%%%%%%%%%%%%%%%%%%%%%%%

%------------------------------------------------------------------------------
%	PACKAGES AND OTHER DOCUMENT CONFIGURATIONS
%------------------------------------------------------------------------------

\documentclass{resume} % Use the custom resume.cls style

\usepackage[left=0.75in,top=0.6in,right=0.75in,bottom=0.6in]{geometry} % Document margins
\usepackage{marvosym}
\usepackage{amsmath}
\usepackage{eso-pic}% http://ctan.org/pkg/eso-pic
\usepackage{lipsum}% http://ctan.org/pkg/lipsum
\usepackage{lastpage}
\usepackage{fancyhdr}
\usepackage{color}
\usepackage{fontawesome}

\pagestyle{fancy}
\fancyhf{}
\renewcommand{\headrulewidth}{0pt}

\rfoot{Page \thepage\ of \pageref{LastPage}}

\newcommand{\sectionspace}{\vspace{3mm}}
\newcommand{\Sergio}{\textbf{S. Garc\'{i}a-Vergara}}
\newcommand{\SuperPop}{Super Pop VR\textsuperscript{TM}\space}
\newcommand{\simplelistentry}[3]{\textit{#1}, {#2} \hfill {#3}}
\newcommand{\fixme}{\textcolor{red}{\textbf{fix me}} \space}
\newcommand{\attention}[1]{\noindent \fixme \textcolor{red}{#1}}

\name{Sergio E. Garc\'{i}a-Vergara} % Your name

\address{\Telefon \space Cell: (787) 383 - 0475}
\address{e\Letter \space sergiodotgarcia@gmail.com \textbullet \faGithub github.com/sgarciav}
\address{\Letter \space 2799 White Oak Ln \textbullet Decatur, GA 30032}


\begin{document}

% add 'US Citizen' at top right corner of first page
\AddToShipoutPictureBG*{%
  \AtPageUpperLeft{%
    \hspace{\paperwidth}%
    \raisebox{-\baselineskip}{%
      \makebox[-25pt][r]{US Citizen}
}}}%


%------------------------------------------------------------------------------
%	OBJECTIVE SECTION
%------------------------------------------------------------------------------

\iffalse
\begin{rSection}{Objective}

  Searching for a part-time software developer position.

\end{rSection}
\fi


%------------------------------------------------------------------------------
%	RESEARCH INTERESTS SECTION
%------------------------------------------------------------------------------

\iffalse
\begin{rSection}{Research Interests}

  Autonomous robotic systems, systems and controls, healthcare robotics,
  assistive technology, human-robot interaction, pattern recognition, and
  machine learning.

\end{rSection}
\fi



%------------------------------------------------------------------------------
%	SUMMARY SECTION
%------------------------------------------------------------------------------

\begin{rSection}{Summary}

  %% More than 11 years of experience in algorithm development for autonomous
  %% robotic systems, software engineering, pattern recognition and machine
  %% learning, and human-robot interaction. Extensive experience conducting system
  %% field tests and user studies, including experiment design and data analysis.

\begin{itemize}
\item More than 11 years of experience in algorithm development for autonomous
  robotic systems, software engineering, pattern recognition and machine
  learning, and human-robot interaction.
\item Extensive experience conducting system field tests and user studies,
  including experiment design and data analysis.
\end{itemize}

\end{rSection}




%------------------------------------------------------------------------------
%	EDUCATION SECTION
%------------------------------------------------------------------------------

\sectionspace
\begin{rSection}{Education}

{\bf Georgia Institute of Technology} \hfill {\bf Atlanta, GA} \\
\textit{Ph.D. in Electrical and Computer Engineering} \hfill May 2017 \\
Dissertation: \textit{Coupling of an Objective and Quantifiable Methodology for Assessing Upper-body Movements with VR Gaming Platforms}

{\bf Georgia Institute of Technology} \hfill {\bf Atlanta, GA} \\
\textit{MS in Electrical and Computer Engineering} \hfill May 2014 \\
Minor: \textit{Computer Science}

{\bf University of Puerto Rico at Mayag\"{u}ez} \hfill {\bf Mayag\"{u}ez, PR} \\
\textit{BS in Electrical Engineering} \hfill June 2011

\end{rSection}




%------------------------------------------------------------------------------
%	SKILLS SECTION
%------------------------------------------------------------------------------

\sectionspace
\begin{rSection}{Skills}

\begin{tabular}{ @{} >{\bfseries}l @{\hspace{6ex}} l }
Programming Languages & Python, C, C++, C\#, Java, Matlab \\

Software Frameworks  & ROS \& ROS 2, TensorFlow, PyTorch, YoloV5, CVAT, FiftyOne \\

% Engineering Software & PSpice, Eclipse, Visual Studio, Simulink, LabView \\

% Operating Systems & Linux, Windows \\

Tools & Docker, CMake, Git, Cygwin, Bash, \LaTeX, Emacs \\

Robotic Platforms & Universal Robots Arms, Dingo, DARwin-OP, Pioneer 3-AT \\
& BlueEagle, DJI S1000 \\ % AmigoBot

Languages & Fully proficient in English and Spanish. (Basic knowledge in German).

\end{tabular}

\end{rSection}




%------------------------------------------------------------------------------
%	RESEARCH EXPERIENCE SECTION
%------------------------------------------------------------------------------

\sectionspace
\begin{rSection}{Work Experience}
%------------------------------------------------

\begin{rSubsection}{CTO \& Co-founder}{October 2020 - present}{RIF Robotics $\vert$ Atlanta, GA}{}

\item
\begin{itemize}
\item Developing a system to autonomously inspect surgical instruments and
  assemble surgical trays to help hospitals reduce patient infections and
  operating room delays with robotics, artificial intelligence, and predictive
  analytics.

\item Responsible for technical development, fundraising, proposal writing, and
  customer interfacing.
\end{itemize}

\end{rSubsection}

%------------------------------------------------

\begin{rSubsection}{Robotics Consultant}{February 2021 - October 2022}{Greenzie $\vert$ Atlanta, GA}{}

\item
\begin{itemize}
\item Provided freelance consulting services to the development team towards
  enhancing the autonomy architecture software of their autonomous lawnmowers.

\item Tasks included improved pipeline for obstacle detection and avoidance,
  improved navigation solutions, and research updating autonomy architecture
  from ROS to ROS 2.
\end{itemize}

\end{rSubsection}

%------------------------------------------------

\begin{rSubsection}{Research Engineer II}{January 2017 - October 2020}{Georgia Tech Research Institute $\vert$ Atlanta, GA}{Supervisor: Dr. Charles Pippin}

\item
\begin{itemize}
\item Developed algorithms for collaborative autonomous systems including, but
  not limited to, task allocation, path planning, and computer vision.

\item Helped develop the lab’s autonomy architecture software and build system.

\item Responsibilities included software and algorithm development, field
  testing autonomous systems, proposal creation, technical reporting, customer
  interfacing, and project management.

\end{itemize}

\end{rSubsection}

%------------------------------------------------

\begin{rSubsection}{Graduate Research Assistant}{May 2012 - December 2016}{Georgia Tech HumAnS Lab $\vert$ Atlanta, GA}{Supervisor: Dr. Ayanna M. Howard}

\item
\begin{itemize}
\item Developed an interactive virtual reality gaming system for rehabilitation
  in the home environment.

\item Developed an objective and quantifiable methodology for evaluating the
  kinematic performance of individuals who have some form of motor skills
  disorder.

\item Developed a pattern recognition algorithm to determine the level of the
  user's kinematic performance such that the virtual reality platform can
  autonomously adapt to the user's needs.

\end{itemize}

\end{rSubsection}

%------------------------------------------------

\iffalse
\begin{rSubsection}{Graduate Research Assistant}{August 2011 - May 2012}{Georgia Tech MRL Lab $\vert$ Atlanta, GA}{Supervisor: Dr. Ronald C. Arkin}

\item
\begin{itemize}

\item Implemented the architecture and support for knowledge sharing across
  heterogeneous robotic agents as part of the MAST (Micro Autonomous Systems
  Technology) project.

\item Designed the conceptual spaces for the different robotic platforms based
  on their respective sensors as a base for the communication and interpretation
  of the acquired data (i.e. vision, laser range finder, thermal, etc).

\end{itemize}

\end{rSubsection}
\fi


%------------------------------------------------
\end{rSection}





%------------------------------------------------------------------------------
%	PUBLICATIONS SECTION
%------------------------------------------------------------------------------

\sectionspace
% \begin{rSection}{Publications and Presentations}
\begin{rSection}{Publications}

%------------------------------------------------
% JOURNALS and BOOK CHAPTERS
\begin{rSubsection}{Journal Publications and Book Chapters}{}{}{}
\item
\begin{enumerate}

\item Y.P. Chen, \Sergio, and A.M. Howard, ``Effect of feedback from a socially
  interactive humanoid robot on reaching kinematics in children with and without
  cerebral palsy: a pilot study,'' \textit{Developmental Neurorehabilitation},
  Vol. 21, No. 8, pp. 490-496, 2018.

%% \item Y.P. Chen, \Sergio, and A.M. Howard, ``Effect of a Home-Based Virtual
%%   Reality Intervention for Children with Cerebral Palsy using \SuperPop
%%   Evaluation Metrics: A Feasibility Study,'' \textit{Rehabilitation Research and
%%     Practice}, 2015.

\item \Sergio, L. Brown, H.W. Park, and A.M. Howard, ``Engaging children in play
  therapy: The coupling of virtual reality games with social robotics,''
  \textit{Technologies of Inclusive Well-Being}, Springer Berlin Heildelberg,
  pp. 139-163, 2014.
\end{enumerate}
\end{rSubsection}



\sectionspace
%------------------------------------------------
% REFEREED CONFERENCE PUBLICATIONS

\begin{rSubsection}{Refereed Conference Publications}{}{}{}
\item
\begin{enumerate}

\item \Sergio, L. Brown, Y.P. Chen, and A.M. Howard, ``Increasing the Efficacy
  of Rehabilitation Protocols for Children via a Robotic Playmate Providing
  Real-time Corrective Feedback,'' \textit{IEEE Conference on Robot and Human
    Interactive Communication (Ro-Man)}, pp. 700-705, 2016.

\item \Sergio, M.M. Serrano, Y.P. Chen, and A.M. Howard, ``Developing a Baseline
  for Upper-body Motor Skill Assessment Using a Robotic Kinematic Model,''
  \textit{IEEE Conference on Robot and Human Interactive Communication
    (Ro-Man)}, pp. 911-916, 2014.

% \item R.C. Arkin, \Sergio, and S.G. Lee, ``Architectural Design and Support for
%   Knowledge Sharing Across Heterogeneous MAST systems,'' \textit{SPIE
%     Conference}, pp. 84070C, 2012.

\end{enumerate}
\end{rSubsection}


%------------------------------------------------
\end{rSection}








%------------------------------------------------------------------------------
%	PATENTS SECTION
%------------------------------------------------------------------------------


\sectionspace

\begin{rSection}{Patents}
%------------------------------------------------

% PATENTS
\begin{rSubsection}{Patents}{}{}{}
\item
\begin{enumerate}
\item R.E. Torres-Mu\~{n}iz, \textbf{S.E. Garc\'{i}a-Vergara},
  B.A. Llorens-Bonilla, D. S\'{a}nchez-Cordero, and M. Lizama, ``Switch-Actuated
  Joystick for Power Wheelchairs'', U.S. Patent 8 622 166 B1, January 7, 2014.
\end{enumerate}

Developed a switch-actuated adapter for joystick controlled wheelchairs such
that individuals with limited mobility can continue making use of their chairs
and avoid spending money on new ones.

\end{rSubsection}
%------------------------------------------------
\end{rSection}






%------------------------------------------------------------------------------
%	REFERENCES SECTION
%------------------------------------------------------------------------------

\iffalse
\sectionspace
\begin{rSection}{References}
Available upon request.
\end{rSection}
\fi


%------------------------------------------------------------------------------

\end{document}

%%% Local Variables:
%%% mode: latex
%%% TeX-master: t
%%% End:
